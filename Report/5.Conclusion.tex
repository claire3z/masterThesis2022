
% \chapter {Conclusion} \label{conclusion}

\section{Discussion}  

%(1 page) The most important results and findings of the work are summarised (not simply a repetition of the structure of the previous chapters!), which new concepts, methods and tools have been developed, which problems can now be solved (more efficiently),

% key contributions
In this thesis, we have presented a text mining tool to extract causality from financial documents. We have also designed a heterogeneous graph-based data model to store the extracted information and to faciliate data exploration. Furthermore, we have demonstrated how this model can be used to assists financial analysts to solve specific, practical problems in investment research.

% causality extraction and practical considerations
After a comprehensive survey of causality extraction techniques, we have decided to adopt a linguistic pattern-based approach, rather than using the latest deep learning models with supervised training. This design choice is based on practical limitations such as avaialblity of training data, cost of annotation, model intepretability, etc. Also taking into consideration is the structure of financial reports and the language styles [TODO] making the linguistic approach possible. Indeed, the causality extraction module has achieved a satisfactory performance level with an overall F1-score of 90\%, albeit the benchmarking dataset is relatively small. This shows that for domain-specific problems, the simpler method might work as well as the most advanced sophisciated methods but with the added practical advantage of saving cost and time.


% clustering
To mimic In order to translate the ... into ... 

To facilitate 


% data exploration and visualization


By systematically extracting causal factors from financial text, and storing them in a graph database, time and cost reduction can be achieved. 

\section{Future Work} 
% (1 page) an outlook on what is to come (e.g. ~What would you do if you had 6 months more time).
